% Created 2018-04-29 Sun 18:32
% Intended LaTeX compiler: pdflatex
\documentclass[presentation]{beamer}
\usepackage[utf8]{inputenc}
\usepackage[T1]{fontenc}
\usepackage{graphicx}
\usepackage{grffile}
\usepackage{longtable}
\usepackage{wrapfig}
\usepackage{rotating}
\usepackage[normalem]{ulem}
\usepackage{amsmath}
\usepackage{textcomp}
\usepackage{amssymb}
\usepackage{capt-of}
\usepackage{hyperref}
\setbeamertemplate{navigation symbols}{}
\AtBeginSection[]{\begin{frame}<beamer>\frametitle{Topic}\tableofcontents[currentsection]\end{frame}}
\usetheme{default}
\author{Ramana Nagasamudram, Khayyam Saleem}
\date{}
\title{Gradual Typing}
\subtitle{An Introduction and an Implementation in MIT/GNU Scheme}
\institute[CS810]{CS810 -- Type Systems for Programming Languages}
\hypersetup{
 pdfauthor={Ramana Nagasamudram, Khayyam Saleem},
 pdftitle={Gradual Typing},
 pdfkeywords={},
 pdfsubject={},
 pdfcreator={Emacs 26.1 (Org mode 9.1.9)}, 
 pdflang={English}}
\begin{document}

\maketitle
\begin{frame}{Outline}
\tableofcontents
\end{frame}



\section{Gradual Typing}
\label{sec:orgdf9c309}
\begin{frame}[label={sec:orga259e2e}]{Concept}
\begin{itemize}
\item Type system developed by Jeremy Siek and Walid Taha in 2006
\item Allows some parts of a program to be dynamically typed and other parts to be statically typed
\begin{itemize}
\item Determined by presence of type annotation added by programmer
\end{itemize}
\end{itemize}
\end{frame}
\begin{frame}[label={sec:orgc925b5e}]{Static Typing}
\begin{itemize}
\item Process of verifying the type safety of a program based on analysis of a program's source code
\item If a program passes a static type checker, then the program is guaranteed to satisfy some set of type safety properties for all possible inputs
\item Type checking completed during compilation process
\end{itemize}



\begin{columns}
\begin{column}{0.5\columnwidth}
\begin{itemize}
\item Pros
\begin{itemize}
\item catches bugs early
\item faster execution
\item improves modularity
\end{itemize}
\end{itemize}
\end{column}


\begin{column}{0.5\columnwidth}
\begin{itemize}
\item Cons
\begin{itemize}
\item makes code more verbose
\item prevents program execution
\end{itemize}
\end{itemize}
\end{column}
\end{columns}
\end{frame}

\begin{frame}[label={sec:org0806ef4}]{Dynamic Typing}
\begin{itemize}
\item Process of type-checking at run-time
\item Associates each runtime object with a \emph{type tag}
\end{itemize}


\begin{columns}
\begin{column}{0.5\columnwidth}
\begin{itemize}
\item Pros
\begin{itemize}
\item offers flexibility
\item doesn't get "in the way" of execution
\item allows for typing based on runtime information
\end{itemize}
\end{itemize}
\end{column}

\begin{column}{0.5\columnwidth}
\begin{itemize}
\item Cons
\begin{itemize}
\item cannot conclusively declare safety
\item errors may lie deep in subroutine calls
\item slower execution
\end{itemize}
\end{itemize}
\end{column}
\end{columns}
\end{frame}

\begin{frame}[label={sec:orge735448}]{Utility of Gradual Typing}
\begin{itemize}
\item Gradual typing allows for type checks at compile-time for type errors in some parts of a program, directed by type annotations.
\item Since it is tough to declare that static typing is universally better or worse than dynamic typing, gradual typing offers the programmer a choice, without requiring a change in language
\end{itemize}
\end{frame}

\begin{frame}[label={sec:orga1eeddf}]{Attempt with Subtyping}
\end{frame}
\begin{frame}[label={sec:orgb5e6399}]{Type Consistency}
\end{frame}
\section{\(\lambda_{\rightarrow}^?\)}
\label{sec:org1741458}
\begin{frame}[label={sec:orgc9834ea}]{Syntax}
\end{frame}
\begin{frame}[label={sec:orga5189c8}]{Typing Rules}
\end{frame}
\begin{frame}[label={sec:orga9b7c6a}]{Relation to simply-typed \(\lambda\)-calculus}
\end{frame}
\begin{frame}[label={sec:orgb84a370}]{Run-time Semantics}
\end{frame}
\section{\(\lambda_{\rightarrow}^{\langle\tau\rangle}\)}
\label{sec:org4edfcdd}
\begin{frame}[label={sec:orgc4b441c}]{First-order example}
\end{frame}
\begin{frame}[label={sec:org39c3398}]{Higher-order example}
\end{frame}
\section{Implementation}
\label{sec:org338c809}
\begin{frame}[label={sec:orgff5b5f2}]{Existing Implementations}
\end{frame}
\begin{frame}[label={sec:org4671385}]{MIT/GNU Scheme Implementation}
\end{frame}
\end{document}
